	%!TEX root = ../../Main.tex
\graphicspath{{Chapters/TestResultater/}}
%-------------------------------------------------------------------------------


\section{Test resultater}

Prototypen til BA-TA systemet er gennemgået test, som er prædefineret af de 3 Use Cases.
Herunder, i tabel \ref{test_resultater}, kan resultatet af de 3 forskellige tests ses og dermed hvorvidt resultatet af testen er godkendt eller ej. Yderligere er der tilknyttet eventuelle bemærkninger til testens forløb og resultater.
%\setlength{\arrayrulewidth}{1mm}
\setlength{\tabcolsep}{18pt}
%\renewcommand{\arraystretch}{1.5}

{
\centering
\label{test_resultater}
\begin{tabular}{ |p{4.2cm}|p{4.2cm}|p{4.2cm}|  }
		\hline
		\multicolumn{3}{|c|}{\textbf{Test resultater af BA-TA prototypen}} \\
		\hline
		Use Case \#& Godkendt eller ikke-godkendt? &Bemærkninger \\
		\hline
		\#1 & Godkendt & Test af Use Case 1 gik som forventet. En fundet BT-enhed kan tilføjes til den godkendte liste. \\
		\hline
		\#2 & Godkendt & Test af Use Case 2 gik som forventet. En tidligere godkendt BT-enhed kan fjernes fra den godkendte liste. \\
		\hline
		\#3 & Godkendt & Test af Use Case 3 gik som forventet. Alle 3 lock/låse states virker. I tilfælde af at en godkendt enhed på listen vælger at skifte navn på den respektive enhed, kan BA-TA på nuværende tidspunkt ikke opdatere til det nye navn i listen over godkendte enheder. \\
		\hline
\end{tabular}
}
