	%!TEX root = ../../Main.tex
\graphicspath{{Chapters/Alternative/}}
%-------------------------------------------------------------------------------


\section{Tilegnelse af viden}


*Kort beskrivelse af hvordan vi har tilegnet og den viden der har været nødvendig for at projektet er i den version den er i på nuværende tidspunkt, samt den viden der ligger bag funktionaliteten i de forskellige blokke/moduler.*

\subsection{Moduler, blokke og protokoller}
  
\textbf{AT-kommandoer} \\
Dele af gruppen har på forhånd været bekendt med AT-kommandoer og deres opbygning og struktur. Denne viden er tilegnet i praktik på 5. semester, hvor der fra en microcontroller blev brugt AT-kommandoer til at initialisere og styre et IoT-device.

\textbf{Bluetooth modulet - HC-05}\\
Tilegnelsen af viden til Bluetooth-SMT-modulet (Surface-Mount-Technology), HC-05, har forummer og generelle internetsøgninger været nødvendige. Da modulet ikke bliver produceret af én bestemt producent har der ikke været et decideret datasheet som gruppen har kunne bruge. Derimod er modulet meget kendt og er brugt af mange mennesker. Det afspejler sig i de mange forskellige blogge gruppen har fundet, hvor modulet er blevet brugt i et projekt. Her har vi kunne hente nok information om modulet og de AT-kommandoer som modulet kan genkende.
Især nedenstående side har været meget brugbar:
http://www.martyncurrey.com/arduino-with-hc-05-bluetooth-module-at-mode/
(Lav til referencer senere)

\subsection{Kode-klasser}
*Her kan evt tilføjes noget om funktioner eller andre principper som bliver brugt i koden, som er viden vi har tilegnet os undervejs*

*** delimetering i Bluetooth.c
strcat
strcpy
strdup ***