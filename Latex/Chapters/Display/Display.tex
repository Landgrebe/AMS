%!TEX root = ../../Main.tex
\graphicspath{{Chapters/Display/}}
%-------------------------------------------------------------------------------


\section{Display driver}
Igennem øvelser til undervisningen, er der blevet opbygget en driver til et grafisk display. Igennem processen til dette projekt er der blevet modificeret og arbejdet videre på selvsamme driver. Cpp filen ligger under TFTdriver, som er bygget op af flere forskellige funktioner. I dette afsnit vil der gives et overblik over de mest essentielle funktioner. For at få en forståelse af selve opbygningen af driveren, henvises til datasheet for controlleren. 
\href{https://blackboard.au.dk/bbcswebdav/pid-1697983-dt-content-rid-3847230_1/courses/BB-Cou-UUVA-73302/BB-Cou-UUVA-65758_ImportedContent_20170106021228/BB-Cou-STADS-UUVA-52360_ImportedContent_20160107025559/LAB/Lab3a%20Graphic%20LCD%20Display/Files%20for%20LAB3a/ILI9341_v1.11.pdf}{ILI9341} \\
For at opbygge en basisforståelse for driveren, vil der først blive forklaret DisplayInit(). Først sætter vi de porte vi skal bruge fra Arduino til hhv. indgange og udgange. Vi har dog valgt ikke at have tilbagemeldinger, og derfor er der ikke initialiseret nogle indgange. Herefter sætter vi RESET, CS, WRX, RDX høje ift figur \ref{fig:Bus_timing}. Der bliver kørt en reset (tjek lige koden med timing, burde være kortere), for at resette displayet, og bagefter sendt fire kommandoer, som kan findes i command list i databladet. Sleep Out, Display On, Pixel format set = 16 bit og Memory Access Control (BGR = 1) (Tjek ift henning hvad dette gør). 



\begin{figure}[H]
	\centering
	\includegraphics[width = 400 pt]{Img/Bus_timing.png}
	\caption{Bus Timing}
	\label{fig:Bus_timing}
\end{figure}

Herunder vil de essentielle funktioner deles op i hvert sin tabel. Flere af funktionerne gør brug af både Writecommand() og Writedata(), som er opbygget udfra Bus Timing figur \ref{fig:Bus_timing}. Derudover bruger flere af funktionerne SetColumnAddress() og SetPageAddress(), som er bygget op fra datasheet 8.2.20. og 8.2.21.: \\

\newpage
\textbf{\Large Included Font:}

\begin{center}
\begin{tabular}{ |l|l|l| }
\hline
\multicolumn{1}{ |c| }{\textbf{character.h}} \\
\hline
Karakter størrelse: 24*24 pixels  \\
Antal karakter: 95\\
\hline

\end{tabular}
\end{center} 
\begin{figure}[H]
	\centering
	\includegraphics[width = 150 pt]{Img/Thedotfactory.png}
	\caption{The Dot Factory karakter}
	\label{fig:Thedotfactory}
\end{figure}

\textbf{\Large Funktions:}

\begin{center}
\begin{tabular}{ |l|l|l| }
\hline
\multicolumn{1}{ |c| }{\textbf{ FillRectangle(StartX,StartY, Width, Height, Red, Green, Blue)}} \\
\hline
Funktion som laver en firkant med den valgte baggrundsfarve  \\
\hline
\textbf{Paramentre:}  \\ StartX: Startværdi på x-aksen \\StartY: Startværdi på y-aksen\\ Width: Bredden på firkanten\\ Height: Højde på firkanten\\ Red: farve(værdi)\\ Green: farve(værdi) \\ Blue: farve(værdi)\\
\hline
\end{tabular}
\end{center} 



\begin{center}
\begin{tabular}{ |l|l|l| }
\hline
\multicolumn{1}{ |c| }{\textbf{ letterhelper(numberletter, startx, starty,Red, Green, Blue)}} \\
\hline
Funktion som bruger input fra Include filen character,\\ til at bestemme hver karakter bredde og længde\\
\hline
\textbf{Paramentre:}  \\numberletter: Bestemmer hvilken karakter der skal hentes\\ Startx: Startværdi på x-aksen \\Starty: Startværdi på y-aksen\\ Red: farve(værdi)\\ Green: farve(værdi) \\ Blue: farve(værdi)\\
\hline
\end{tabular}
\end{center}

\begin{center}
\begin{tabular}{ |l|l|l| }
\hline
\multicolumn{1}{ |c| }{\textbf{ drawXBitmap( bitmap[],length,count,startx,starty, letter, modulus,Red, Green, Blue)}} \\
\hline
Funktion som står for at skrive til hver bit, med værdier fra letterhelper()\\
\hline
\textbf{Paramentre:}  \\bitmap[]: Henter en byte fra character.h\\ length: Fortæller funktionen, hvor bred karakteren der skal skrives er\\ count: hvor mange bytes funktionen skal køre igennem for at lave hele karakteren\\ Startx: Startværdi på x-aksen \\Starty: Startværdi på y-aksen\\ Red: farve(værdi)\\ Green: farve(værdi) \\ Blue: farve(værdi)\\
\textbf{Note:} Funktionen sletter gamle karakterer, da en ny smallere karakter end forrige stadig\\ vil forblive på displayet\\

\hline
\end{tabular}
\end{center}  

\begin{center}
\begin{tabular}{ |l|l|l| }
\hline
\multicolumn{1}{ |c| }{\textbf{ drawletters(str[],startx, starty,Red, Green, Blue)}} \\
\hline
Funktion som modtager en en streng, og konverterer ascii værdien til \\den rette værdi ift character.h \\
\hline
\textbf{Paramentre:}  \\str[]: Modtager en streng\\  Startx: Startværdi på x-aksen \\Starty: Startværdi på y-aksen\\ Red: farve(værdi)\\ Green: farve(værdi) \\ Blue: farve(værdi)\\
\\

\hline
\end{tabular}
\end{center}  

\begin{center}
\begin{tabular}{ |l|l|l| }
\hline
\multicolumn{1}{ |c| }{\textbf{ drawNumber(number,startx, starty,Red, Green, Blue)}} \\
\hline
Funktion som modtager en en integer, og konverterer ascii værdien til \\den rette værdi ift character.h \\
\hline
\textbf{Paramentre:}  \\number: Modtager et interger tal\\  Startx: Startværdi på x-aksen \\Starty: Startværdi på y-aksen\\ Red: farve(værdi)\\ Green: farve(værdi) \\ Blue: farve(værdi)\\
\\

\hline
\end{tabular}
\end{center}  

\begin{center}
\begin{tabular}{ |l|l|l| }
\hline
\multicolumn{1}{ |c| }{\textbf{ setBaggroundPixel(int red, int green, int blue)}} \\
\hline
Funktion som sætter baggrundsfarven af tekst og tal\\
\hline
\textbf{Paramentre:}  \\ Red: farve(værdi)\\ Green: farve(værdi) \\ Blue: farve(værdi)\\
\\

\hline
\end{tabular}
\end{center}  


