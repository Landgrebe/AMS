%!TEX root = ../../Main.tex
\graphicspath{{Chapters/Display/}}
%-------------------------------------------------------------------------------


\section{Bluetooth modul}

Bluetooth-modulet der bliver brugt i BA-TA systemet er af typen HC-05. Denne type-version understøtter nemlig mulighed for at tage rollen som både Master/Slave i et Bluetooth Master/Slave forhold imellem to parrende Bluetooth-enheden. Bluetooth modulet er sammensat med et breakout- board i bunden af modulet, hvilket giver nem adgang til de brugbare pins på selve Bluetooth-modulet.

\subsection{Protokol}
Der kommunikeres til systemets Bluetooth-modul igennem en UART-forbindelse. Gruppens implementerede UART-driver på Arduino’en gør det muligt, at Arduinoen og Bluetooth-modulet kan kommunikere sammen vha. AT-kommandoer. 
AT-kommandoerne (Attention-commands) som Bluetooth-modulet understøtter kan bruges til generel opsætning af modulet, sætte Master/Slave roller, scanne for nærtliggende Bluetooth-enheder, samt at spørge om navn på fundne Bluetooth-enheders adresser. AT-kommandoerne er en general standard som bruges til at sende instruktioner til et modul og dermed er de opbygget på samme måde for hvert modul der bruger dem. På en etableret UART forbindelse vil man kunne sende en ”AT+kommando” til modulet, hvorefter modulet vil udføre kommandoen, og evt sende et svar tilbage, samt et OK, hvilket indikerer at beskeder var succesfuldt modtaget af modulet. Det samme er gældende for HC-05 Bluetooth modulet som indgår i BA-TA systemet.
Eksempler på brugte AT-kommandoer til Bluetooth-modulet i BA-TA:

\begin{itemize}
	\item ”AT+INIT” – Initialisering af Bluetooth-modulets Bluetooth SPP profil
	\item “AT+INQM,0,5,4” – Sætter scanner parameter. I dette tilfælde at der maksimum skal findes 5 Bluetooth enheder og at timeout på en scanning af Bluetooth enheder er 4*1.28 sekunder.
	\item ”AT+RNAME?<indsæt,Adresse,Her>” – Bluetooth-modulet spørger om navnet på adressen til den respektive indsætte adresse.
\end{itemize}

En succesfuld AT-kommando vil altid få Bluetooth-modulet til at returnere et OK. Såfremt der eksempelvis bliver spurgt om et navn på en respektiv Bluetooth-enheds adresse, så vil Bluetooth-modulet også returnere den fundne Bluetooth-enheds navn og et OK til sidst. 

**INDSÆT EKSEMPEL FRA UART-FORBINDELSE TIL MODULET HER**

\subsection{Implementering}

Da AT-kommandoerne og retur-beskederne er opbygget med samme struktur, har gruppen med fordel taget brug af dette til, at kunne udvikle en Bluetooth-kommunikations-driver, som netop udnytter denne gentagende og genkendelige proces. 


**INDSÆT EKSEMPEL FRA KODEN HER EVT. – SOM VISER HVORDAN VI HÅNDTERER SVAR FRA MODULET
