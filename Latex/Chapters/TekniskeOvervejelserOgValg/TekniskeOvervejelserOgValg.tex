	%!TEX root = ../../Main.tex
\graphicspath{{Chapters/Alternative/}}
%-------------------------------------------------------------------------------


\section{Tekniske overvejelse og valg}



\subsection{Fordele og ulemper}
Da projektets omfang og opbygning har været meget frit, er der også blevet lavet nogle valgt og fravalg i process fasen. Dette vil vi give et overblik over i dette afsnit, forklare fordele og ulmeper ved hvert modul vi har valgt, og driverne dertil. Herunder gives et overblik over de emner gruppen mener har været mest essentielle ift fordele og ulemper af de forekellige moduler. 
\newline
\newline
\textbf{TFT Display} \\*
Da omfanget for projektet inkluderede et display, gjorde gruppen et valg om at bruge TJC-9341-032 som display, og \href{https://blackboard.au.dk/bbcswebdav/pid-1697983-dt-content-rid-3847230_1/courses/BB-Cou-UUVA-73302/BB-Cou-UUVA-65758_ImportedContent_20170106021228/BB-Cou-STADS-UUVA-52360_ImportedContent_20160107025559/LAB/Lab3a%20Graphic%20LCD%20Display/Files%20for%20LAB3a/ILI9341_v1.11.pdf}{ILI9341} 
som display controller. Dette \href{https://blackboard.au.dk/bbcswebdav/pid-1697983-dt-content-rid-3847230_1/courses/BB-Cou-UUVA-73302/BB-Cou-UUVA-65758_ImportedContent_20170106021228/BB-Cou-STADS-UUVA-52360_ImportedContent_20160107025559/LAB/Lab3a%20Graphic%20LCD%20Display/Files%20for%20LAB3a/ILI9341_v1.11.pdf}{ILI9341} 
er valgt på baggrund af, gruppen har abejdet med dette display modul tidligere, derudover var dette også tilgængeligt.
Til selve display'et er der valgt kun at kunne skrive til display'et og ikke læse fra det. Fordelen ved kun at sende data til displayet, er at det gør opsætningen meget nemmere for udvikleren. \\
Fordelen ved at intialisere driveren til at kunne modtage data fra displayet, ville være at programstrukturen, kunne spørge skærmen hvilket display, der var på skærmen nu. Dette vil sikre at programstrukturen altid ved hvilket frame der vises på displayet. Ulempen ved den måde gruppen har intialiseret driveren på, er at microcontroller skal holde styr på hvilken frame, der vises på skærmen. Dette gør det mere udfordrende for udvikleren, og derved gør at koden kommer til at fylde mere. Hvis driveren skulle laves på en memory kritisk microcontroller, ville det være en fordel at tilføje skærmen at kunne læse hvilket frame der står på skærmen. \\
Vi har ikke brugt meget tid på at undersøge alternative display. Der blev valgt at et Alphanumeric display, som tidligere var bearbejdet, ikke villde opfylde de krav vi havde til displayet, og da controlleren shielded ITDB02 
, havde en Touch controller del, blev Alphanumeric display'et valgt fra.
\newline
\newline
\textbf{Gør brug af globale variabler} \\*
Igennem de forskellige drivere gøres der brug af globale variabler. Dette er gjort for at simplificire udviklingen af driverne. Denne metode at programmere på er dog ikke god programmerings-skik, og hvis gruppen havde afsat mere tid til udviklingen, ville det første være at erstatte disse globale variabler, med get() og set() metoder. 
\newline
\newline
\textbf{Bluetooth Modul} \\
De første uger af udviklingsperioden var der i gruppen større problemer med det Bluetooth modul der først var blevet lånt af \href{https://stockmanager.ase.au.dk/}{Embedded Stock} til implementering i systemet. Efter en uges tid med troubleshooting på hvorfor den ikke kunne sættes op til at søge efter andre enheder, vidste det sig at det udleverede modul ikke var det rigtige modul som gruppen havde bestilt. Der var blevet bestilt et HC-05 modul, men istedet udleveret et HC-06 modul. Forskellen kan ikke ses med det blotte øje, da modulerne fysisk er ens og den eneste reele forskel er firmwaren den er blevet sendt afsted med fra fabrikken. HC-06 firmwaren er meget mere simpelt bygget op og understøtter derfor ikke samme antal avancerede AT-kommandoer. Dermed kan den ikke initialiseres til at kunne søge efter andre enheder og dermed lave inquiries.

Herefter blev der lånt et anerledes HC-05 med et større breakout-board, hvilket efter nogle få timers debugging og troubleshooting måtte konkluderes at være i stykker. Det lykkedes os dog igennem Embedded Stock at få lånt et nyt HC-05 modul af anerledes breakout-board. Dette var funktionelt og endelig kunne gruppen begynde på selve implementeringen af Bluetooth-modulet i systemet, som inkluderer alle de funktioner den skal kunne understøtte fra Arduino'en. Der var dog også snak om hvorvidt gruppen skulle bestille et dyrere Bluetooth-modul på Amazon, men i og med vi først fik prøvet det andet af, kunne gruppen konstatere at det fint ville fungerere i sammenhæng med resten af systemet. Yderlige havde gruppen på dette tidspunkt ikke behov for flere forsinkelser med Bluetooth-moduler og derfor var der ikke incitament for at skulle vente på at posten ville nå frem.

Den benyttede Bluetooth-modul i systemet, som kan ses på figur \ref{fig:bluetooth_modul}, er dog anderledes bygget op på breakout-boardet, hvilket betyder, at der har skulle træffes nogle valg alt efter, hvordan gruppen ville bruge Bluetooth-modulet i systemet. Bluetooth-modulet har 2 muligheder: den kan styres af kommandoer igennem Bluetooth-protokolen OTA (over-the-air) af enheder den har oprettet en Bluetooth forbindelse til og så kan man styre den ved brug af UART forbindelse og AT-kommandoer til dens RX/TX ben.

